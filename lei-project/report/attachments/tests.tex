\tocSec{\textbf{Testes}}
\label{testsAttachments}

Devido ao tempo limitado não foi possível realizar testes a todo o projeto, porém foi possível testar o seu funcionamento recorrendo ao \textbf{Jest} e o \textit{\glslinkUnder{packages}{package}} incluído no \textbf{React} para testes. Desta forma foi possível realizar alguns testes unitários a dois componentes, realizando algumas validações através das propriedades que estes recebem. Além destes dois testes, foi possível testar como realizar o preenchimento de \textit{inputs}, simulando a interação por parte do utilizador.

\begin{longlisting}
	\inputminted[]{jsx}{code/tests/button.test.tsx}
	\caption{Testes realizados ao componente \texttt{Button}}
\end{longlisting}

\begin{longlisting}
	\inputminted[highlightlines={17},highlightcolor=yellow!25]{jsx}{code/tests/status-badge.test.tsx}
	\caption{Testes realizados ao componente \texttt{tatus Badge}}
\end{longlisting}

O resultado dos testes apresentados acima seria:

\figureFrame{.75}{passed-tests.png}{Execução dos testes com sucesso}

Se no caso do teste realizado ao componente \textbf{\texttt{Status Badge}}, na linha 17 (que se encontra sombreada), o valor \texttt{"Ativo"} fosse alterado, o teste iria dar erro, visto este não conter assim a \textit{class} \texttt{active-status}. Para esse erro seria apresentado a mensagem presente na figura que se segue.

\figureFrame{.55}{error-tests.png}{Execução dos testes com erro no componente \textbf{\texttt{Status Badge}}}

Outro ponto que é necessário ter em consideração e, tal como é possível analisar nos testes apresentados, a ``pesquisa'' do elemento é feito pelo \texttt{getByTestId}, sendo que para tal é necessário adicionar no código do componente o \mintinline{html}{data-testid="<id>"}:

\begin{longlisting}
	\begin{minted}[highlightlines={3},highlightcolor=yellow!25]{jsx}
	return (
		<div
			data-testid="status-badge"
			{...attributes}
			className={`${styles["root"] || ""} ${className || ""} ${
				styles[`${statusColor}-status`]
			}`}
		>
			{status}
		</div>
	);
	\end{minted}
	\caption{Código com o \texttt{data-testid} definido no componente \textbf{\texttt{Status Badge}}}
\end{longlisting}

O excerto de código que se segue, apresenta um teste onde é realizada a verificação do valor do componente \textbf{\texttt{TextInput}} no ecrã de inicio de sessão, ou seja, numa primeira fase é inserido o valor no campo de texto (no caso no campo \textit{email}) e posteriormente é validado o seu valor, simulando assim a interação que é realizada pelo utilizador.

\begin{longlisting}
	\inputminted[]{jsx}{code/tests/insert-text-input.test.tsx}
	\caption{Teste com preenchimento de \textit{inputs}}
\end{longlisting}

Seria ainda possível realizar testes com dados falsos, ou chamados de \textit{mock data}, recorrendo para tal ao \textit{\glslinkUnder{packages}{package} \texttt{@testing-library/react}}. Para tal, seria simulado um pedido a \textbf{\glsShortUnder{api}}, no caso do projeto seria através do \textbf{Apollo Client}, onde é esperada uma resposta.

Na documentação do \textbf{Apollo Client}\footnote{\textbf{Documentação oficial e outras referências:} \cite{apolloTestsDocs,mockingApolloTesting,apolloTest,mockingData,mockingGraphql}} é possível encontrar exemplos de como utilizar as ferramentas já incluídas na instalação do mesmo, recorrendo para tal ao \texttt{MockedProvider}, um elemento que recebe uma lista de \texttt{mocks}, ou seja, respostas e dados factícios.

No caso do inicio de sessão existiriam duas possibilidades, o inicio de sessão com sucesso, onde era necessário testar o redirecionamento por parte do \textbf{React Router}, ou então, o inicio de sessão inválido, onde é apresentada uma mensagem de erro ao utilizador.

A pergunta pode ser, mas qual a razão pela qual usar dados fictícios? Num caso real de um inicio de sessão onde é utilizado \textit{email} e \textit{password}, basta o utilizador alterar a \textit{password} para os testes falharem, porém a funcionalidade de inicio de sessão continua a funcionar. Com os \textit{mock data} isto não acontece, visto simular um pedido fictício à \textbf{\glsShortUnder{api}}.
