\tocSubSubSec{\textit{Avatar}}
\label{avatarComp}

O componente \textbf{\textit{Avatar}} é encontrado nos mais diversos ecrãs do projeto, contando com várias possiblidades de apresentação, tal como é possível analisar nas figuras que se seguem.

\begin{minipage}{0.55\textwidth}
	\figureFrame{1}{avatar-athlete-profile.jpeg}{Componente \textbf{Avatar} na página de perfil do atleta}
\end{minipage}
\begin{minipage}{0.35\textwidth}
	\figureFrame{1}{avatar-athletes-mosaic.jpeg}{Componente \textbf{Avatar} na vista em mosaico na lista de atletas}
\end{minipage}

\figureFrame{1}{avatar-athletes-list.jpeg}{Componente \textbf{Avatar} na lista de atletas}

De uma forma resumida, o componente \textbf{Avatar} pode ser apenas a imagem do atleta, a imagem do atleta com o número de notificações, a imagem de atleta e nome do alteta, imagem do atleta e identificador, ou então, a junção de todas estas possibilidades.

\begin{longlisting}
	\begin{minted}[]{jsx}
		const Avatar = ({
			className,
			avatar,
			notifications,
			label,
			...attributes
		}: IAvatarProps) => {
			return (
				<div
					className={`${styles["root"] || ""} ${className || ""}`}
					{...attributes}
				>
					<div
						className={styles["avatar"]}
						style={{
							backgroundImage: avatar ? `url(${avatar})` : undefined
						}}
					>
						{label && <span className={styles["label"]}>{label}</span>}

						{notifications && (
							<span className={styles["badge"]}>{notifications}</span>
						)}
					</div>
				</div>
			);
		};

		export default Avatar;
	\end{minted}

	\caption{Código desenvolvido para o componente \textbf{Avatar}}
\end{longlisting}

Como é possível analisar pelo excerto de código anterior, o componente \textbf{Avatar} recebe determinadas propriedades, podendo algumas ser ou não recebidas, permitindo assim criar as várias vertentes apresentadas nas figuras.