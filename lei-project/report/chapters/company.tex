\tocChap{Apresentação da Entidade de Acolhimento}

\begin{minipage}[t!]{0.35\textwidth}
	\figureFrame{.5}{jimmy-boys.png}{\textbf{Jimmy Boys} \textemdash~Icon}
\end{minipage}
\begin{minipage}[t!]{.65\textwidth}
	\minipagerestore

	A \textbf{Jimmy Boys} é uma empresa que opera no ramo do desenvolvimento de \textit{software} desde 2012. A \textbf{Jimmy Boys} desenvolve tanto os próprios \textit{softwares}, bem como em \textit{outsourcing} para outras empresas.

	A \textbf{Jimmy Boys} opera tanto em \textit{\glslinkUnder{frontend}{front-end}}, \textit{\glslinkUnder{backend}{back-end}} e \textit{mobile}, realizando projetos nas mais diversas tecnologias, como \textbf{React}, \textbf{GraphQL}, \textbf{Rust}, \textbf{Flutter}, entre outras. Além destas tecnologias, realiza ainda projetos de \glsShortUnder{ui} e \glsShortUnder{ux}.
\end{minipage}

\vspace{0.2cm}

\begin{flushright}
	\begin{quotebox50}
		``Along the way, we have been working with different technologies and different business needs. This helped us grow and prepared us for more demanding projects.''

		\tcblower

		\href{https://www.jimmyboys.pt/about-us}{Jimmy Boys}
	\end{quotebox50}
\end{flushright}

\vspace{0.2cm}

\textit{Outsourcing} consiste na contratação de recursos a outra empresa. Por exemplo, quando uma empresa não possui um departamento de \textit{marketing}, recorre a uma empresa desta área para realizar esse serviço em nome desta empresa.

Ao trabalhar em \textit{outsourcing}, a \textbf{Jimmy Boys} além de disponibilizar os seus colaboradores para a realização do projeto em questão, promove ainda \textit{workshops} dentro da empresa, com o objetivo de integrar a equipa da empresa no projeto, explorando temas como boas práticas no desenvolvimento ou, como criar um projeto em determinada tecnologia.

Abaixo seguem as principais ligações da empresa.

\begin{itemize}
	\item \textbf{\href{https://www.linkedin.com/company/jimmy-boys/}{LinkedIn}};
	\item \textbf{\href{https://jimmyboys.pt}{\textit{Website}}}
\end{itemize}

\newpage