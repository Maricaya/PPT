\section{Resultados}

A primeira interação com o projeto começou com a criação de componentes \textbf{React}, porém utilizando \textbf{JavaScript}, o que pouco tempo depois, levou à migração de todo o código já produzido para \textbf{TypeScript}. O \textbf{TypeScript} é atualmente quase que um requisito obrigatório, tendo também a sua parte de garantia de qualidade do código, visto que os erros são mais facilmente detectados devido à tipagem necessária.

Posteriormente, após a migração de ambas as vertentes (\textit{backoffice} e \textit{frontoffice}) foram realizados em primeiro lugar os componentes com mais utilização no projeto\footnote{Ver \hyperrefUnder{genericComponents}{anexos {\footnotesize(página 62 a 70)}}.}, componentes estes que numa interação futura seriam colocados numa biblioteca de componentes uma vez que são utilizados em ambas as vertentes. Após estes componentes criados, o processo de desenvolvimento passou por criar páginas, nesta fase com dados estáticos, que por sua vez, originavam a criação de outros componentes.

Uma das grandes aprendizagens na realização deste projeto foi o uso de \glslinkUnder{sprite}{sprites} \textbf{\glsShortUnder{svg}} para armazenar todos os ícones necessários para o projeto, sendo criado um componente \textbf{React} que iria aceder a cada ícone colocado nesta \glslinkUnder{sprite}{sprite} através do \texttt{id} a este associado.

No capítulo referente aos resultados é possível analisar com mais detalhe todo o desenvolvimento realizado, apresentando figuras relativas ao mesmo, bem como os componentes \textbf{React} desenvolvidos para as necessidades que cada página apresentava.