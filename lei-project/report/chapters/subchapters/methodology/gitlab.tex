\subsection{GitLab}

A primeira ferramenta utilizada para controlar as tarefas existentes para o projeto foi o \textbf{GitLab}, criando para tal \textit{issues}, sendo apresentadas numa \textit{board}. Posteriormente, era através destas \textit{issues} também criado um \textit{merge request}, ficando assim associados.

\figureFrame{1}{gitlab.jpeg}{Board utilizada no \textbf{GitLab}}

A imagem anterior apresenta a \textit{board} utilizada, bem como as colunas existentes para as vários etapas de desenvolvimento de uma \textit{issue}.